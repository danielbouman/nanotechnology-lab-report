% !TeX root = main.tex
\section*{Lithographic system}
The basic working of optical contact lithography is illustrated in figure \ref{fig:contact-litho}. Light emitted from a source is collimated using a condenser lens (or an array of lenses). The collimated light travels through the transparent\todo[inline]{is transparant wel het juiste woord hiervoor?} parts of the mask and illuminates the resist area directly below these areas. 
\begin{figure}[H]
	\centering
	\resizebox{0.7\linewidth}{!}{% !TeX root = main.tex
\begin{tikzpicture}
    % Basic outline
    \draw (0,0) -- (1,1);
\end{tikzpicture}
}
	\caption{Simplified working of contact lithography, with 1. light source, 2. condenser lens(es), 3. glass/quarts plate with mask, 4. photographic resist, 5. substrate. Dashed lines indicate lightrays.}
	\label{fig:contact-litho}
\end{figure} For this experiment a MJB-3 mask aligner from Karl Suss Microtec with a mercury-vapor lamp was used. However, this particular model is designed for deep ultraviolet (DUV) while the used resist is designed for exposure to the I-line and H-line (365.4~nm and 404.7~nm respectively) of the lamp or near ultraviolet (NUV). To prevent the DUV light from exposing the resist and damaging it, the mask is clamped to a glass plate which blocks light with $\lambda \lesssim 300$~nm.\todo[inline]{misschien erbij zetten waarom we dan dit apparaat gebruikt hebben? of klint het lulligis het onnodig om er bij te zetten dat ze hier te arm zijn voor meer goede apparaten?}.
For the mask a design was made featuring several patterns on different scales. The design of the mask can be found in the appendix.


\section*{Substrate and resist}
Patterns were written on a series of square cut silicon substrates of approximately 15 mm by 15 mm. For both the negative and positive tone samples, AZ~5214 E\footnote{AZ~5214 E is a high resolution image reversal resist produced by MicroChemicals} is used as photoresist. Since the optics of the MJB-3 is designed for DUV light, a high percentage of the intensity of the NUV is absorbed \todo[inline]{is het mogelijk om te berekenen/schatten hoeveel procent ongeveer?}. Thus the exposure times of standard recipes designed by the Kavli Nanolab facility are not sufficient. The recipes are modified to accommodate the decrease in light intensity. Some initial guessing was needed and different exposure times are investigated.

\subsection*{Positive resist}
As a positive tone resist, AZ~5214 E is used. To promote adhesion to the Si substrate, HMDS (hexamethyldisilazane) is applied first as a primer. The primer is deposited by hand on top of a silicon wafer and spun at 4000 RPM, after which it is then baked on a hot plate at 200$^{\circ}$~C for two minutes. The resist is spun at the same speed as the primer, which should result in a layer thickness of $1.40 \mu$m\todo[inline]{referentie naar recipe. ik had wat problemen met referentie plaatsen met een url erin..}. The resist is baked on a hot plate at 90$^{\circ}$~C for one minute.

After the resist is deposited, the sample is exposed in the mask aligner for 1, 2, 2.5, 3, 3.5, and 4 minutes. After exposure the sample is developed for 60 seconds in MF-321 \footnote{MF-321 developer is mainly composed of water and tetramethylammonium hydroxide, it is produced by Microposit.} after which the development is stopped by rinsing the sample another 60 seconds in purified water.

\subsection*{Negative resist}
The recipe used for the positive tone is adjusted to get a negative tone. Up to the exposure, the negative resist recipe is the same as that for the positive tone. AZ~5214E contains a special cross-linking agent which becomes active at temperatures above 110$^{\circ}$~C where the resist has been exposed. The cross-linking agent causes the individual molecules to bond, creating an almost insoluble, non-photoreactive substance. This allows the AZ~5214E to also be used as a negative resist.

Using the same mask as for the positive exposure, the sample is illuminated for a period 1 \todo[inline]{check time} minutes. After this first illumination the sample is baked in an oven at ... \todo[inline]{temp?} $^\circ$~C for 45\todo[inline]{er lijkt 42 te staan op ons blad..?} seconds. During this time cross-links are formed in the areas that were exposed during the first illumination. During baking the sample lies on an aluminium slab inside the oven, which prevents large temperature drops when the oven door is opened and ensures good heat transfer to the sample. After baking, the entire sample is exposed (flood exposure) for a period of either 1,2,3,4 \todo[inline]{all times} minutes. During this time the polymer chains in the areas of the resist that were not exposed during the first illumination, are cut up into smaller chains. During development these smaller chains are dissolved in the development solution \todo[inline]{name}, while the cross-linked areas remain on the substrate. After development, the sample is rinsed with purified water as was done with the positive resist samples\cite{productdatasheet}.
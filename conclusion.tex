% !TeX root = main.tex
\section*{Conclusion}
In this report we presented the effects of using AZ 5214 E as both positive and negative tone resist for creating microstructures using a MJB-3 mask aligner. Several different exposure times were used to approach optimal times. For positive tone resist, the best exposure time was estimated to be 2~min, resulting in a MFS of 2-10 $\mu$m. For negative tone resist, the optimal exposure time was estimated to be around 0.4~min, although there were no samples made to confirm this. The MFS of the negative tone sample with an initial exposure 0.5~min. and a flood exposure time of 4~min., was found to be 20-40~$\mu$m, which is well above the diffraction limited optimum (MFS = 0.81~$\mu$m).This shows that 2.0~min. for the positive tone resist is close to optimal, while 0.5 min. for negative tone resist is far from optimal. 

The edges of both the positive and negative patterns suffer from over- and undercut, respectively. This can be seen near the edges of the patterns by the increased contrast in the SEM images and the dark areas in the microscope images. While some undercut is expected for the negative tone, the amount of overcut for the positive tone is higher than desired. The overcut is typical of low dosages, so it is expected that higher resolutions will be attainable with higher dosages.
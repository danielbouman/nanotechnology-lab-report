% !TeX root = main.tex
\section{Mask design}
\begin{figure}[H]
    \centering
    \includegraphics[trim=0mm 0.12mm 0.17mm 0mm, clip=true,width=\linewidth]{figures/litho_design.pdf}
    \captionof{figure}{Used lithography design}
    \label{fig:litho_design}
\end{figure}

The design for the mask was made to enable easy detection of the attained resolution with certain patterns. This was done by taking patterns specifically designed to test the resolution and/or by taking patterns that are common in the semiconductor industry. Starting from the top left and moving clockwise,

\begin{itemize}
    \item Checkerboard
    \item Single line surround by multiple lines
    \item Outward radiating lines
    \item Internally touching circles
    \item Shapes with OPC
    \item Slightly non-parallel lines
    \item Koch snowflake
\end{itemize}

The checkerboard pattern is a widely used pattern in micro- and nanofabrication checkerboard pattern, allowing easy detection of over- and underexposure. The next pattern is a much seen pattern in nanofabrication, and is important for analyzing the effect of interference when multiple lines are close together. The outward radiating lines have the feature that the distance between the lines becomes smaller towards the center, making it possible to see at what length scale the lines are still separable. This principle also holds for the internally touching circles. In the top right there is a square with an (approximate) optical proximity correction (OPC) pattern. The Koch snowflake has (in the idealized case) infinite detail, allowing the comparison between the different length scales of the snowflake to see where the resolution becomes too low to see the details. Note that this version has only five iterations, since infinite detail would be unpractical.

\section{Inspection images}\label{ap:inspec_img}
 \todo[inline]{ik krijg de titel niet op dezelfde pagina als de plaatjes..}
 % \todo[inline]{Optical microscope images of mask}
 \begin{figure*}[!t]
     \centering
     \begin{subfigure}[t]{0.24\linewidth}
     	\centering
  	\resizebox{\linewidth}{!}{\includegraphics{data/mask/TS_05_20_13_50_44.jpg}}
  	\caption{Single line surrounded by multiple lines pattern at several length scales.}
  	\label{fig:TS_05_20_13_50_44}
  \end{subfigure}
     \hfill
     \begin{subfigure}[t]{0.24\linewidth}
  	\centering
  	\resizebox{\linewidth}{!}{\includegraphics{data/mask/TS_05_20_13_51_16.jpg}}
  	\caption{Inverse of the previous pattern.}
  	\label{fig:TS_05_20_13_51_16}
  \end{subfigure}
     \hfill
     \begin{subfigure}[t]{0.24\linewidth}
  	\centering
  	\resizebox{\linewidth}{!}{\includegraphics{data/mask/TS_05_20_13_55_57.jpg}}
  	\caption{Zoom in on the previous image.}
  	\label{fig:TS_05_20_13_55_57}
  \end{subfigure}
\hfill
     \begin{subfigure}[t]{0.24\linewidth}
  	\centering
  	\resizebox{\linewidth}{!}{\includegraphics{data/mask/TS_05_20_13_51_30.jpg}}
  	\caption{Checkerboard pattern at several length scales.}
  	\label{fig:TS_05_20_13_51_30}
  \end{subfigure}
 \end{figure*}


 \begin{figure*}[!t]
     \centering
     \begin{subfigure}[t]{0.24\linewidth}
  	\centering
  	\resizebox{\linewidth}{!}{\includegraphics{data/mask/TS_05_20_13_52_44.jpg}}
  	\caption{Outward radiating lines}
  	\label{fig:TS_05_20_13_52_44}
  \end{subfigure}
\hfill
     \begin{subfigure}[t]{0.24\linewidth}
  	\centering
  	\resizebox{\linewidth}{!}{\includegraphics{data/mask/TS_05_20_13_53_38.jpg}}
  	\caption{Koch snowflake.}
  	\label{fig:TS_05_20_13_53_38}
  \end{subfigure}
     \hfill
     \begin{subfigure}[t]{0.24\linewidth}
  	\centering
  	\resizebox{\linewidth}{!}{\includegraphics{data/mask/TS_05_20_13_54_28.jpg}}
  	\caption{Internally touching circles.}
  	\label{fig:TS_05_20_13_54_28}
  \end{subfigure}
     \hfill
     \begin{subfigure}[t]{0.24\linewidth}
  	\centering
  	\resizebox{\linewidth}{!}{\includegraphics{data/mask/TS_05_20_13_54_51.jpg}}
  	\caption{Squares with OPC patterns.}
  	\label{fig:TS_05_20_13_54_51}
  \end{subfigure}
\caption{Microscope images of some parts of the designed mask.}
 \end{figure*}



 \begin{figure*}[!t]
     \centering
     \begin{subfigure}[t]{0.32\linewidth}
 	\resizebox{\linewidth}{!}{\includegraphics{data/b3d1.jpg}}
  	\caption{Exposure time of 1.0 minutes}
  	\label{fig:b3d1}
 \end{subfigure}
\hfill
     \begin{subfigure}[t]{0.32\linewidth}
 	\resizebox{\linewidth}{!}{\includegraphics{data/b3a1.jpg}}
  	\caption{Exposure time of 2.0 minutes}
  	\label{fig:b3a1}
 \end{subfigure}
\hfill
     \begin{subfigure}[t]{0.32\linewidth}
 	\resizebox{\linewidth}{!}{\includegraphics{data/b3e1.jpg}}
  	\caption{Exposure time of 2.5 minutes}
  	\label{fig:b3e1}
 \end{subfigure}\\
     \begin{subfigure}[t]{0.32\linewidth}
 	\resizebox{\linewidth}{!}{\includegraphics{data/b3b2.jpg}}
  	\caption{Exposure time of 3.0 minutes}
  	\label{fig:b3b2}
 \end{subfigure}
\hfill
     \begin{subfigure}[t]{0.32\linewidth}
 	\resizebox{\linewidth}{!}{\includegraphics{data/b3c1.jpg}}
  	\caption{Exposure time of 3.5 minutes}
  	\label{fig:b3c1}
 \end{subfigure}
\hfill
     \begin{subfigure}[t]{0.32\linewidth}
 	\resizebox{\linewidth}{!}{\includegraphics{data/b3f1.jpg}}
  	\caption{Exposure time of 4.0 minutes}
  	\label{fig:b3f1}
 \end{subfigure}
\caption{Microscope images of the positive tone samples for several exposure times.}
\end{figure*}



% % \todo[inline]{Optical microscope images of samples}
% \begin{figure*}[!t]
%     \centering
%     \begin{subfigure}[t]{0.32\linewidth}
%  	\resizebox{\linewidth}{!}{\includegraphics{data/b3b1.jpg}}
%  	\caption{b3b1}
%  	\label{fig:b3b1}
%  \end{subfigure}
% \hfill
%     \begin{subfigure}[t]{0.32\linewidth}
%  	\centering
%  	\resizebox{\linewidth}{!}{\includegraphics{data/b3b2.jpg}}
%  	\caption{b3b2}
%  	\label{fig:b3b2}
%  \end{subfigure}
% \hfill
%     \begin{subfigure}[t]{0.32\linewidth}
%  	\centering
%  	\resizebox{\linewidth}{!}{\includegraphics{data/b3b3.jpg}}
%  	\caption{b3b3}
%  	\label{fig:b3b3}
%  \end{subfigure}\\
%     \centering
%     \begin{subfigure}[t]{0.32\linewidth}
%  	\resizebox{\linewidth}{!}{\includegraphics{data/b3c1.jpg}}
%  	\caption{b3c1}
%  	\label{fig:b3c1}
%  \end{subfigure}
% \hspace{10mm}
% \begin{subfigure}[t]{0.32\linewidth}
%  	\centering
%  	\resizebox{\linewidth}{!}{\includegraphics{data/b3f1.jpg}}
%  	\caption{b3f1}
%  	\label{fig:b3f1}
% \end{subfigure}
%  \end{figure*}
%







 % \todo[inline]{SEM}
 \begin{figure*}[!t]
     \centering
     \begin{subfigure}[t]{0.32\linewidth}
  	\centering
  	\resizebox{\linewidth}{!}{\includegraphics{data/sem/b3a1_q01.jpg}}
  	\caption{Structure size of $\sim$6 $\mu$m. There is a small defect in the line pattern on the left side, which was most likely caused by improper handling of the sample.}
  	\label{fig:b2d1_q1}
  \end{subfigure}
 \hfill
     \begin{subfigure}[t]{0.32\linewidth}
  	\centering
  	\resizebox{\linewidth}{!}{\includegraphics{data/sem/b3a2_q02.jpg}}
  	\caption{Structure size of $\sim$2 $\mu$m. The pattern loses its structure near the center due to interference effects.}
  	\label{fig:b2d2_q2}
  \end{subfigure}
 \hfill
     \begin{subfigure}[t]{0.32\linewidth}
  	\centering
  	\resizebox{\linewidth}{!}{\includegraphics{data/sem/b3a3_q03.jpg}}
  	\caption{Structure size of $\sim$1 $\mu$m. Almost the whole pattern is dominated by interference at these length scales.}
  	\label{fig:b2d3_q3}
  \end{subfigure}
\caption{Outward radiating lines pattern for positive tone. For the structure size the width of the line at the edge of the pattern is taken as guideline.}
 \end{figure*}

 \begin{figure*}[!t]
     \centering
     \begin{subfigure}[t]{0.24\linewidth}
  	\centering
  	\resizebox{\linewidth}{!}{\includegraphics{data/sem/b3a11_q12.jpg}}
  	\caption{Structure size of $\sim$12 $\mu$m.}
  	\label{fig:b2d12_q12}
  \end{subfigure}
\hfill
     \begin{subfigure}[t]{0.24\linewidth}
  	\centering
  	\resizebox{\linewidth}{!}{\includegraphics{data/sem/b3a11_q13.jpg}}
  	\caption{Zoom of the previous picture. Rounding off the corners can be seen. These could in principle be corrected by using OPC.}
  	\label{fig:b2d13_q13}
  \end{subfigure}
 %\hfill
 %    \begin{subfigure}[t]{0.24\linewidth}
 % 	\centering
 % 	\resizebox{\linewidth}{!}{\includegraphics{data/sem/b3a11_q14.jpg}}
 % 	\caption{SEM}
 % 	\label{fig:b2d14_q14}
 %  \end{subfigure}
 \hfill
     \begin{subfigure}[t]{0.24\linewidth}
  	\centering
  	\resizebox{\linewidth}{!}{\includegraphics{data/sem/b3a11_q15.jpg}}
  	\caption{Structure size of $\sim$1.5 $\mu$m. The rounding of the corners has become more pronounced.}
  	\label{fig:b2d15_q15}
   \end{subfigure}
 \hfill
     \begin{subfigure}[t]{0.24\linewidth}
  	\centering
  	\resizebox{\linewidth}{!}{\includegraphics{data/sem/b3a11_q16.jpg}}
  	\caption{Severe interference effects start to occur at the bends for structure sizes $\sim$1$\mu$m.}
  	\label{fig:b2d16_q16}
  \end{subfigure}
\caption{Single line surrounded by multiple lines for positive tone. For the structure size the width of the line is taken as guideline.}
 \end{figure*}

 \begin{figure*}[!t]
     \centering
%     \begin{subfigure}[t]{0.24\linewidth}
%  	\centering
%  	\resizebox{\linewidth}{!}{\includegraphics{data/sem/b3a11_q17.jpg}}
%  	\caption{SEM}
%  	\label{fig:b2d17_q17}
%  \end{subfigure}
% \hfill
 %    \begin{subfigure}[t]{0.24\linewidth}
 % 	\centering
 % 	\resizebox{\linewidth}{!}{\includegraphics{data/sem/b3a11_q18.jpg}}
 % 	\caption{SEM}
 % 	\label{fig:b2d18_q18}
 % \end{subfigure}
 %\hfill
     \begin{subfigure}[t]{0.32\linewidth}
  	\centering
  	\resizebox{\linewidth}{!}{\includegraphics{data/sem/b3a11_q19.jpg}}
  	\caption{Structure size of $\sim$50 $\mu$m.}
  	\label{fig:b2d19_q19}
  \end{subfigure}
 \hfill
     \begin{subfigure}[t]{0.32\linewidth}
  	\centering
  	\resizebox{\linewidth}{!}{\includegraphics{data/sem/b3a11_q20.jpg}}
  	\caption{Structure size of $\sim$30 $\mu$m. In comparison with the previous image, a significant drop in detail can be seen at the eges.}
  	\label{fig:b2d20_q20}
  \end{subfigure}
\hfill
     \begin{subfigure}[t]{0.32\linewidth}
  	\resizebox{\linewidth}{!}{\includegraphics{data/sem/b3a11_q21.jpg}}
  	\caption{Structure size of $\sim$4 $\mu$m. All fine details have been lost.}
  	\label{fig:b2d21_q21}
  \end{subfigure}
\caption{Koch snowflake for positive tone. For the structure size the width of one of the bulbs is taken as guideline.}
 \end{figure*}









 \begin{figure*}[!t]
     \centering
     \begin{subfigure}[t]{0.24\linewidth}
  	\centering
  	\resizebox{\linewidth}{!}{\includegraphics{data/sem/b2d22_q22.jpg}}
  	\caption{Structure size of $\sim$20 $\mu$m. Already at these length scales rounding of the corners can be seen. The pattern disappears in the center due to the overexposure.}
  	\label{fig:b2d22_q22}
  \end{subfigure}
 \hfill
     \begin{subfigure}[t]{0.24\linewidth}
  	\centering
  	\resizebox{\linewidth}{!}{\includegraphics{data/sem/b2d23_q23.jpg}}
  	\caption{Zoom in of the previous picture, to display how the pattern tapers of due to overexposure.}
  	\label{fig:b2d23_q23}
  \end{subfigure}
 \hfill
     \begin{subfigure}[t]{0.24\linewidth}
  	\centering
  	\resizebox{\linewidth}{!}{\includegraphics{data/sem/b2d24_q24.jpg}}
  	\caption{Structure size of $\sim$5 $\mu$m. The pattern is almost completely gone. Even after the tapering of the pattern, the photoactivation of the resist is still visible.}
  	\label{fig:b2d24_q24}
  \end{subfigure}
\hfill
     \begin{subfigure}[t]{0.24\linewidth}
  	\resizebox{\linewidth}{!}{\includegraphics{data/sem/b2d25_q25.jpg}}
  	\caption{Zoom in on the previous picture.}
  	\label{fig:b2d25_q25}
  \end{subfigure}
 \hfill
\caption{Outward radiating lines pattern for negative tone. For the structure size the width of the line at the edge of the pattern is taken as guideline.}
 \end{figure*}

 \begin{figure*}[!t]
     \centering
     \begin{subfigure}[t]{0.24\linewidth}
  	\centering
  	\resizebox{\linewidth}{!}{\includegraphics{data/sem/b2d30_q30.jpg}}
  	\caption{Structure size of $\sim$10 $\mu$m.}
  	\label{fig:b2d30_q30}
  \end{subfigure}
 \hfill
     \begin{subfigure}[t]{0.24\linewidth}
  	\centering
  	\resizebox{\linewidth}{!}{\includegraphics{data/sem/b2d31_q31.jpg}}
  	\caption{Zoom in of the previous image. Rounding of the corners can be seen.}
  	\label{fig:b2d31_q31}
  \end{subfigure}
 \hfill
     \begin{subfigure}[t]{0.24\linewidth}
  	\centering
  	\resizebox{\linewidth}{!}{\includegraphics{data/sem/b2d31_q32.jpg}}
  	\caption{Inverse of the two previous images.}
  	\label{fig:b2d31_q32}
  \end{subfigure}
\hfill
     \begin{subfigure}[t]{0.24\linewidth}
  	\centering
  	\resizebox{\linewidth}{!}{\includegraphics{data/sem/b2d33_q34.jpg}}
  	\caption{No significant differences can be seen between the inverses of the pattern.}
  	\label{fig:b2d33_q34}
 \end{subfigure}
\caption{Single line surrounded by multiple lines for negative tone. For the structure size the width of the line is taken as guideline. For structure sizes smaller than $\sim$10$\mu$m the pattern loses all detail.}
 \end{figure*}



 \begin{figure*}[!t]
     \centering
%     \begin{subfigure}[t]{0.24\linewidth}
%  	\centering
%  	\resizebox{\linewidth}{!}{\includegraphics{data/sem/b2d33_q35.jpg}}
%  	\caption{SEM}
%  	\label{fig:b2d33_q35}
% \end{subfigure}
% \hfill
     \begin{subfigure}[t]{0.32\linewidth}
  	\centering
  	\resizebox{\linewidth}{!}{\includegraphics{data/sem/b2d36_q36.jpg}}
  	\caption{Structure size of $\sim$20$\mu$m. There is already significant loss of detail at these structure sizes.}
  	\label{fig:b2d36_q36}
 \end{subfigure}
\hfill
     \begin{subfigure}[t]{0.32\linewidth}
  	\resizebox{\linewidth}{!}{\includegraphics{data/sem/b2d36_q37.jpg}}
  	\caption{Structure size of $\sim$10$\mu$m. All fine detail is lost.}
  	\label{fig:b2d36_q37}
 \end{subfigure}
 \hfill
     \begin{subfigure}[t]{0.32\linewidth}
  	\centering
  	\resizebox{\linewidth}{!}{\includegraphics{data/sem/b2d38_q38.jpg}}
  	\caption{Complete loss of detail for structure size $\sim$1$\mu$m.}
  	\label{fig:b2d38_q38}
 \end{subfigure}
% \hfill
%     \begin{subfigure}[t]{0.24\linewidth}
%  	\centering
%  	\resizebox{\linewidth}{!}{\includegraphics{data/sem/b2d39_q39.jpg}}
%  	\caption{SEM}
%  	\label{fig:b2d39_q39}
% \end{subfigure}
% \hfill
%     \begin{subfigure}[t]{0.24\linewidth}
%  	\centering
%  	\resizebox{\linewidth}{!}{\includegraphics{data/sem/b2d40_q40.jpg}}
%  	\caption{SEM}
%  	\label{fig:b2d40_q40}
% \end{subfigure}
\caption{Koch snowflake for negative tone. For the structure size the width of one of the bulbs is taken as guideline.}
  \end{figure*}



% \begin{figure*}[!t]
%     \centering
%     \begin{subfigure}[t]{0.32\linewidth}
% 	\resizebox{\linewidth}{!}{\includegraphics{data/b2i1.jpg}}
%  	\caption{Exposure time of 0.1 minutes}
%  	\label{fig:b2i1}
% \end{subfigure}
%\hfill
%     \begin{subfigure}[t]{0.32\linewidth}
% 	\resizebox{\linewidth}{!}{\includegraphics{data/b2h1.jpg}}
%  	\caption{Exposure time of 0.2 minutes}
%  	\label{fig:b2h1}
% \end{subfigure}
%\hfill
%     \begin{subfigure}[t]{0.32\linewidth}
% 	\resizebox{\linewidth}{!}{\includegraphics{data/b2d1.jpg}}
%  	\caption{Exposure time of 0.5 minutes}
%  	\label{fig:b2d1}
% \end{subfigure}
%\caption{Microscope images of the negative tone samples for several exposure times.}
%\end{figure*}


%
%  \begin{figure*}[!t]
%  	\centering
%  	\resizebox{0.33.\linewidth}{!}{\includegraphics{data/sem/b2d41_q41.jpg}}
%  	\caption{test}
%  	\label{fig:b2d41_q41}
%  \end{figure*}
 %
 %\begin{figure*}[!t]
 %    \centering
 %    \begin{subfigure}[t]{0.24\linewidth}
 %	\resizebox{\linewidth}{!}{\includegraphics{data/sem/b3a1_q01.jpg}}
 %	\caption{SEM}
 %	\label{fig:b2d1_q1}
 %\end{subfigure}
 %\hfill
 %    \begin{subfigure}[t]{0.24\linewidth}
 %	\centering
 %	\resizebox{\linewidth}{!}{\includegraphics{data/sem/b3a2_q02.jpg}}
 %	\caption{SEM}
 %	\label{fig:b2d2_q2}
 %\end{subfigure}
 %\hfill
 %    \begin{subfigure}[t]{0.24\linewidth}
 %	\centering
 %	\resizebox{\linewidth}{!}{\includegraphics{data/sem/b3a3_q03.jpg}}
 %	\caption{SEM}
 %	\label{fig:b2d3_q3}
 %\end{subfigure}
 %    \begin{subfigure}[t]{0.24\linewidth}
 %	\centering
 %	\resizebox{\linewidth}{!}{\includegraphics{data/sem/b3a10_q10.jpg}}
 %	\caption{SEM}
 %	\label{fig:b2d10_q10}
 %\end{subfigure}
 %\end{figure*}

% \begin{figure*}[!t]
% \centering
%     \begin{subfigure}[t]{0.32\linewidth}
% 	\centering
% 	\resizebox{\linewidth}{!}{\includegraphics{data/sem/b3a4_q04.jpg}}
% 	\caption{Structure size of $\sim$32 $\mu$m.}
% 	\label{fig:b2d4_q4}
% \end{subfigure}
% \hfill
%     \begin{subfigure}[t]{0.32\linewidth}
% 	\centering
% 	\resizebox{\linewidth}{!}{\includegraphics{data/sem/b3a6_q06.jpg}}
% 	\caption{Structure size of $\sim$10 $\mu$m.}
% 	\label{fig:b2d6_q6}
% \end{subfigure}
% \hfill
%     \begin{subfigure}[t]{0.32\linewidth}
% 	\centering
% 	\resizebox{\linewidth}{!}{\includegraphics{data/sem/b3a8_q08.jpg}}
% 	\caption{Structure size of $\sim$3 $\mu$m.}
% 	\label{fig:b2d8_q8}
% \end{subfigure}
% \end{figure*}

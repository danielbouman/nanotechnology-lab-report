% !TeX root = main.tex
\section{Lithography design}
\begin{Figure}
    \centering
    \includegraphics[trim=0mm 0.12mm 0.17mm 0mm, clip=true,width=\linewidth]{figures/litho_design.pdf}
    \captionof{figure}{Used lithography design}
    \label{fig:distancea10}
\end{Figure} \todo[inline]{Explain features and perpose of the design}

Starting from the top left and moving clockwise,

\begin{itemize}
\item Checkerboard
\item Single line surround by multiple lines
\item Outward radiating lines
\item Internally touching circles
\item Optical Proximity Correction (OPC)
\item Slightly non-parallel lines
\item Kock snowflake
\end{itemize}

The checkerboard pattern is a widely used pattern in micro- and nanofabrication
checkerboard pattern, allowing easy detection of over- and underexposure.
The next pattern is a much seen pattern in nanofabrication, and is important for analyzing the effect of interference when multiple lines are close together.
The outward radiating lines have the feature that the distance between the
lines becomes smaller towards the center, making it possible to see at what length scale the lines are still separable. This principle also holds for the
internally touching circles. An
inverted version of this pattern is placed below the original.
Top right we have a square with an eyeballed optical proximity correction (OPC) pattern. After
a certain scale we expect that these patterns will start to resemble squares considerably well.
The next pattern consists of two parts. One bent line pattern, and another which uses an
eyeballed OPC-pattern. We expect that after a certain scale, the pattern with the OPC will better
resemble the intended shape than the pattern without OPC.
We then have two lines that are attached at the top, but have a small angle between them.
The point after which the two lines becomes indistinguishable is again a measure for how good the
resolution is.
Finally we have a Koch snowflake. Since the Koch snowflake has (in the idealized case) infinite
detail, we can compare the different length scales of the snowflake to see where the resolution
becomes too low to see the details. Note that this version has only five iterations, since infinite
detail would be unpractical.